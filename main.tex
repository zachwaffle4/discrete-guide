\documentclass{report}
\usepackage[letterpaper, margin=1in]{geometry}
\usepackage{graphicx} % Required for inserting images
\usepackage{amssymb}
\usepackage{amsmath}
\usepackage{mathtools}
\usepackage{float}

\DeclarePairedDelimiter{\card}{\lvert}{\rvert}
\DeclarePairedDelimiter{\makeset}{\{}{\}}

\DeclareMathOperator{\comb}{C}
\DeclareMathOperator{\perm}{P}

\newcommand{\powerset}{\mathcal{P}}
\newcommand{\universal}{\mathbb{U}}

\title{A Brief Guide To Discrete Structures}
\author{Zach Harel}

\begin{document}

\maketitle

\chapter{Logic}

\section{Propositional Logic}

A \textbf{proposition} is a statement that is either true or false.

\subsection{Basic Logical Operators}

Everything in propositional logic can be created using only NOT, AND, and OR.

\subsubsection{$\neg$: Negation (NOT)}
The negation of $p$ ($\neg p$) is true when $p$ is false, and false when $p$ is true.

\subsubsection{$\land$: Conjunction (AND)}
$p \land q$ is true only when both $p$ and $q$ are true.

\subsubsection{$\lor$: Disjunction (OR)}
$p \lor q$ is true when at least one of $p$ or $q$ is true.

\subsection{More Logical Operators}

These operators are commonly used in both mathematics and computer science due to their utility. However, they are simply combinations of the three basic operators.

\subsubsection{$\implies$: Implication (IF-THEN)}
$p \implies q$ is false only when $p$ is true and $q$ is false.

\begin{itemize}
    \item $p$ is the \textbf{hypothesis} (or antecedent)
    \item $q$ is the \textbf{conclusion} (or consequent)
\end{itemize}

\begin{equation}
    p \implies q \equiv \lnot p \lor q
\end{equation}

\subsubsection{$\oplus$: Exclusive Disjunction (XOR)}

$p \oplus q$ is true when either $p$ is true or $q$ is true, but not both.

\begin{equation}
    p \oplus q \equiv (p \lor q) \land \lnot(p \land q)
\end{equation}

\subsection{Logical Equivalence}

Two propositions are \textbf{logically equivalent} ($\equiv$) if they have the same truth value for all possible truth assignments.

There are many laws of logical equivalence; you can Google them, but here are some commonly used ones.

\subsubsection{De Morgan's Laws}
\begin{align}
    \neg(p \land q) &\equiv \neg p \lor \neg q \\
    \neg(p \lor q) &\equiv \neg p \land \neg q
\end{align}

\subsubsection{Double Negation}
\begin{equation}
    \neg(\neg p) \equiv p
\end{equation}

\subsubsection{Contrapositive}
\begin{equation}
    (p \implies q) \equiv (\neg q \implies \neg p)
\end{equation}

\section{First-Order Logic}

First-order logic extends propositional logic with \textbf{predicates} and \textbf{quantifiers}.

\subsection{Predicates}

A \textbf{predicate} is a statement containing variables that becomes a proposition when variables are assigned values.

Example: $P(x)$: ``$x$ is even'' becomes a proposition when $x$ is specified.

\subsection{Quantifiers}

\subsubsection{$\forall$: Universal Quantifier (FOR ALL)}
$\forall x \, P(x)$ means $P(x)$ is true for every value of $x$ in the domain.

\subsubsection{$\exists$: Existential Quantifier (THERE EXISTS)}
$\exists x \, P(x)$ means there is at least one value of $x$ in the domain for which $P(x)$ is true.

\subsection{Negating Quantifiers}

\begin{align}
    \neg(\forall x \, P(x)) &\equiv \exists x \, \neg P(x) \\
    \neg(\exists x \, P(x)) &\equiv \forall x \, \neg P(x)
\end{align}

\subsection{Multiple Quantifiers}

Order matters! $\forall x \, \exists y \, P(x, y)$ is different from $\exists y \, \forall x \, P(x, y)$.

Example: Let $P(x, y)$: ``$x < y$''
\begin{itemize}
    \item $\forall x \, \exists y \, P(x, y)$: For every number, there exists a larger number (TRUE)
    \item $\exists y \, \forall x \, P(x, y)$: There exists a number larger than all numbers (FALSE)
\end{itemize}

\chapter{Representation of Numbers}

\section{Numeric Bases}

A number system with base $b$ uses digits $0$ through $b-1$ to represent values. The rightmost digit represents $b^0$, the next represents $b^1$, and so on.

\subsection{Common Bases}

\begin{itemize}
    \item \textbf{Binary} (base 2): Uses digits 0, 1
    \item \textbf{Octal} (base 8): Uses digits 0-7
    \item \textbf{Decimal} (base 10): Uses digits 0-9
    \item \textbf{Hexadecimal} (base 16): Uses digits 0-9, A-F (where A=10, B=11, ..., F=15)
\end{itemize}

\subsection{Converting From Base $b$ to Decimal}

To convert a number in base $b$ to decimal, multiply each digit by its positional value and sum.

\begin{equation}
    (d_n d_{n-1} \ldots d_1 d_0)_b = d_n \cdot b^n + d_{n-1} \cdot b^{n-1} + \ldots + d_1 \cdot b^1 + d_0 \cdot b^0
\end{equation}

\textbf{Example:} $(1011)_2 = 1 \cdot 2^3 + 0 \cdot 2^2 + 1 \cdot 2^1 + 1 \cdot 2^0 = 8 + 0 + 2 + 1 = (11)_{10}$

\subsection{Converting From Decimal to Base $b$}

Repeatedly divide by $b$ and record remainders from bottom to top.

\textbf{Example:} Convert $(13)_{10}$ to binary:
\begin{align*}
    13 \div 2 &= 6 \text{ remainder } 1 \\
    6 \div 2 &= 3 \text{ remainder } 0 \\
    3 \div 2 &= 1 \text{ remainder } 1 \\
    1 \div 2 &= 0 \text{ remainder } 1
\end{align*}

Reading remainders from bottom to top: $(13)_{10} = (1101)_2$

\subsection{Converting Between Non-Decimal Bases}

Convert through decimal as an intermediate step, or use direct conversion for related bases (e.g., binary $\leftrightarrow$ hexadecimal).

\textbf{Binary to Hexadecimal:} Group binary digits in sets of 4 (starting from right), convert each group to its hex equivalent.

\textbf{Example:} $(11010110)_2 = (1101)(0110)_2 = (D6)_{16}$

\section{Signed and Unsigned Numbers}

\subsection{Unsigned Representation}

All bits represent magnitude. An $n$-bit unsigned number can represent values from $0$ to $2^n - 1$.

\textbf{Example:} With 8 bits, unsigned range is $0$ to $255$.

\subsection{Signed Representation}

Multiple methods exist to represent signed integers:

\subsubsection{Sign-Magnitude}

The leftmost bit is the sign bit (0 for positive, 1 for negative), remaining bits represent magnitude.

\textbf{Problems:} Two representations of zero ($+0$ and $-0$), arithmetic is complex.

\subsubsection{One's Complement}

Negative numbers are formed by flipping all bits of the positive representation.

\textbf{Problems:} Still has two zeros, arithmetic requires end-around carry.

\section{Two's Complement}

Two's complement is the standard representation for signed integers in modern computers.

\subsection{Definition}

For an $n$-bit number:
\begin{itemize}
    \item If the leftmost bit is 0, the number is positive (same as unsigned)
    \item If the leftmost bit is 1, the number is negative
\end{itemize}

An $n$-bit two's complement number can represent values from $-2^{n-1}$ to $2^{n-1} - 1$.

\subsection{Converting to Two's Complement}

To represent $-x$ in two's complement:
\begin{enumerate}
    \item Write the binary representation of $x$
    \item Flip all bits (one's complement)
    \item Add 1
\end{enumerate}

\textbf{Example:} Find $-5$ in 8-bit two's complement:
\begin{align*}
    5 &= (00000101)_2 \\
    \text{Flip bits: } &= (11111010)_2 \\
    \text{Add 1: } &= (11111011)_2
\end{align*}

Therefore, $-5 = (11111011)_2$ in 8-bit two's complement.

\subsection{Converting From Two's Complement}

If the leftmost bit is 1 (negative number), apply the same process: flip all bits and add 1.

\textbf{Example:} What is $(11111011)_2$ in decimal?
\begin{align*}
    \text{Flip bits: } &= (00000100)_2 \\
    \text{Add 1: } &= (00000101)_2 = 5 \\
    \text{Since original was negative: } &= -5
\end{align*}

\subsection{Properties of Two's Complement}

\begin{itemize}
    \item Only one representation of zero: $(00...0)_2$
    \item Range for $n$ bits: $-2^{n-1}$ to $2^{n-1} - 1$
    \item The negative of a number equals its two's complement
    \item The leftmost bit has weight $-2^{n-1}$
\end{itemize}

\section{Arithmetic in Two's Complement}

\subsection{Addition}

Add numbers normally using binary addition. Discard any carry out of the leftmost bit.

\textbf{Example:} $5 + (-3) = 2$ in 8-bit two's complement:
\begin{align*}
    00000101 \text{ (5)} \\
    + \, 11111101 \text{ (-3)} \\
    \hline
    1\,00000010 \text{ (discard carry, result = 2)}
\end{align*}

\subsection{Subtraction}

To compute $a - b$, add $a$ and the two's complement of $b$: $a + (-b)$.

\textbf{Example:} $7 - 5 = 7 + (-5)$:
\begin{align*}
    00000111 \text{ (7)} \\
    + \, 11111011 \text{ (-5)} \\
    \hline
    1\,00000010 \text{ (result = 2)}
\end{align*}

\subsection{Overflow Detection}

Overflow occurs when the result exceeds the representable range.

\textbf{Overflow conditions:}
\begin{itemize}
    \item Adding two positive numbers yields a negative result
    \item Adding two negative numbers yields a positive result
    \item Equivalent: carry into the sign bit $\neq$ carry out of the sign bit
\end{itemize}

\textbf{Example:} $127 + 1$ in 8-bit (overflow):
\begin{align*}
    01111111 \text{ (127)} \\
    + \, 00000001 \text{ (1)} \\
    \hline
    10000000 \text{ (-128, overflow!)}
\end{align*}

\subsection{Sign Extension}

To represent an $n$-bit two's complement number with more bits, copy the sign bit into all new leftmost positions.

\textbf{Example:} Extend $(-5)$ from 8-bit to 16-bit:
\begin{align*}
    \text{8-bit: } &11111011 \\
    \text{16-bit: } &1111111111111011
\end{align*}

\chapter{Sets}

\section{What is a Set?}

A set is simply an unordered collection of unique items.

\section{Default Sets}

\subsection{$\universal$: The Universal Set}
Contains every element in our domain (must be defined for each problem/set of problems).

\subsection{$\varnothing$: The Empty Set} 
\{\}; Has no elements.

\section{Set Operations}

\subsection{=: Equality}

Two sets are equal if all of their elements are the same.

\begin{equation}
    (A = B) \iff (\forall x (x \in A \iff x \in B))
\end{equation}

\subsection{$\card{S}$: Cardinality}

The cardinality of set $S$ ($\vert S \vert$) is the number of elements in set $S$.

\begin{equation}
    \card{\makeset{1, 2, 3}} = 3
\end{equation}

\subsection{$\overline{S}$: Complement}

The complement of a set $S$ ($\overline{S}$) is the set of every element (of the universal set) that is not in $S$.

For example, if $\universal = \makeset{1, 2, 3, 4, 5}$ and $S = \makeset{1, 4}$, $\overline{S} = \makeset{2, 3, 5}$.

\begin{equation}
    \vert \overline{S} \vert = \vert \universal \vert - \vert S \vert
\end{equation}

\subsection{$\cap$: Intersection}

The intersection of sets $A$ and $B$ ($A \cap B$) contains every element in both sets.

\begin{equation}
    A \cap B = \{x | x \in A \land x \in B\} 
\end{equation}

\subsection{$\cup$: Union}

The union of sets $A$ and $B$ ($A \cup B$) contains every element in either set (with no repeats).

\begin{align}
    A \cup B &= \{x \vert x \in A \lor x \in B\} \\
    \card{A \cup B} &= \card{A} + \card{B} + \card{A \cap B}
\end{align}

\subsection{$\subseteq$: Subset}

If $A$ is a subset of $B$ ($A \subseteq B$), all elements of $A$ are also elements of $B$.

\begin{equation}
    A \subseteq B \iff \forall x (x \in A) \implies (x \in B)
\end{equation}

Any set is a subset of itself.

\begin{equation}
    A \subseteq A
\end{equation}

The subset definition can also be used to define set equality.

\begin{equation}
    A = B \iff (A \subseteq B) \land (B \subseteq A)
\end{equation}

If $A$ is a \emph{proper} (also called strict) subset of $B$ ($A \subset B)$,
all elements of $A$ are in $B$, but $A$ is not $B$.

\begin{equation}
    A \subset B \iff (A \subseteq B) \land (A  \not= B)
\end{equation}

If $A$ is a subset of $B$, $B$ is a superset of $A$; if $A$ is a proper subset of $B$, $B$ is a proper superset of $A$.

\begin{align}
    A \subseteq B &\iff B \supseteq A \\
    A \subset B &\iff B \supset A
\end{align}

\subsection{$-$: Difference}

The difference of sets $A$ and $B$ ($A$ - $B$) is every element of $A$ that is not in $B$.

\begin{align}
    A - B &= A \cap \overline{B} \\
    \card{A - B} &= \card{A \cap \overline{B}}
\end{align}

Set difference can also be used to define a set's absolute complement (the previously defined complement):

\begin{equation}
    \overline{S} = \universal - S
\end{equation}

The difference of sets $A$ and $B$ ($A - B$) is sometimes called the relative complement of B with respect to A ($B \backslash A$). The absolute complement of a set $S$  ($\overline{S}) $is simply its relative complement with respect to the universal set ($\universal \backslash {S}$).

\begin{equation}
    A - B = A \backslash B
\end{equation}

The \emph{symmetric} difference of sets $A$ and $B$ ($A \Delta B)$ is every element of one set that is not in the other set.

\begin{equation}
    A \Delta B = (A - B) \cup (B - A)
\end{equation}

\section{Set Functions}

\subsection{Powerset}
\begin{itemize}
    \item input: set
    \item output: set of sets
\end{itemize}

\begin{equation}
    \powerset(S) = \{A \vert A \subseteq S\}
\end{equation}

\subsection{Cartesian Product}
\begin{itemize}
    \item input: two sets
    \item output: set of ordered pairs
\end{itemize}

\begin{equation}
    A \times B = \{(a, b) | a \in A \wedge b \in B \}
\end{equation}

If $A \cap B = \varnothing$, $A$ and $B$ are \textbf{disjoint}.

\chapter{Counting}

\begin{table}[H]
    \centering
    \begin{tabular}{ccc}
         &  No Repetition& Yes Repetition\\
         Order Matters &  Permutation $\perm(n, k)$& $n^k$\\
         Order Does Not Matter &  Combination $\comb(n, k)$ & Stars \& Bars\\
    \end{tabular}
    \caption{Which Formula To Use}
    \label{tab:placeholder}
\end{table}

\section{Counting We've Seen}

\begin{align*}
    \text{let } A &= \makeset{1, 2, 3} \text{ and } B = \makeset{3, 4}
    \\ &\therefore \\
    A \times B &= \makeset{(1, 3), (1, 4), (2, 3), (2, 4), (3, 3), (3, 4)}
\end{align*}


\begin{equation}
    \card{A \times B} = \card{A} \cdot \card{B}
\end{equation}

\begin{equation}
    \card{\powerset(S)} = 2^{\card{S}}
\end{equation}

\section{Product Rule: Order Matters}

\subsection{Two Separate Tasks}
\begin{itemize}
    \item one task in $n$ ways
    \item one task in $m$ ways
\end{itemize}

\begin{equation}
    \text{There are } n \cdot m \text{ ways to do task 1 \textbf{and} task 2.}
\end{equation}

\subsection{One Task Multiple Ways}
When repetition is okay, we have $k$ tasks and $n$ choices per task; there are $n^k$ ways to do it.

When repetition is not okay, we have $k$ tasks and $n$ choices for the first task, $n-1$ for the second task, etc. For this, we use permutations: $\perm(n, k)$.

\begin{align}
    \text{$\perm(n, k)$} &= \text{$k$ permutations of $n$ objects is an ordering of $k$ of the objects.} \\
    \perm(n, k) &= \frac{n!}{(n-k)!} \text{; } \perm(n, n) = n!
\end{align}

\section{Sum Rule: Order Does Not Matter}

\subsection{Two Tasks}
\begin{itemize}
    \item one task in $n$ ways
    \item one task in $m$ ways
\end{itemize}

\begin{equation}
    \text{There are } n + m \text{ ways to do task 1 \textbf{or} task 2.}
\end{equation}

\subsection{One Task Multiple Ways}
When repetition is okay, see the Stars and Bars subsection!

When repetition is not okay, we have $k$ tasks and $n$ choices for the first task, $n-1$ for the second task, etc. For this, we use combinations: $\comb(n, k)$.

\begin{align}
    \text{$\comb(n, k)$} &= \text{$k$ combinations of $n$ objects is a set of $k$ of the objects.} \\
    \comb(n, k) &= \frac{n!}{k!(n-k)!} \text{; } \comb(n, n) = 1
\end{align}

\subsection{Stars and Bars}

\begin{enumerate}
    \item Order doesn't matter
    \item Repetition is allowed
\end{enumerate}

How many ways are there to put $n$ indistinguishable balls into $n$ distinguishable bins?

This is secretly a combination problem! 

\begin{align}
    n_{\text{combination}} &= n_{\text{stars and bars}} + k_{\text{stars and bars}} + 1 \\
    k_{\text{combination}} &= k_{\text{stars and bars}} -1
\end{align}

This means that the answer to our initial question is simply
$\comb(n + k + 1, k - 1)!$

\section{The Pigeonhole Principle}

The bridge between Counting and Probability!
We are putting objects in boxes.

If there are 200 people in a room, it is guaranteed that at least 17 of them share a birth month. It is not guaranteed (it is likely) that even one person was born in October (or any other month, for that matter). 

\chapter{Probability}

\end{document}
