\documentclass{report}
\usepackage[letterpaper, margin=1in]{geometry}
\usepackage{graphicx} % Required for inserting images
\usepackage{amssymb}
\usepackage{amsmath}
\usepackage{float}

\title{CS 1800 Notes}
\author{Zach Harel}

\begin{document}

\maketitle

\chapter{Logic}

\section{Propositional Logic}

A \textbf{proposition} is a statement that is either true or false.

\subsection{Basic Logical Operators}

Everything in propositional logic can be created using only NOT, AND, and OR.

\subsubsection{$\neg$: Negation (NOT)}
The negation of $p$ ($\neg p$) is true when $p$ is false, and false when $p$ is true.

\subsubsection{$\land$: Conjunction (AND)}
$p \land q$ is true only when both $p$ and $q$ are true.

\subsubsection{$\lor$: Disjunction (OR)}
$p \lor q$ is true when at least one of $p$ or $q$ is true.

\subsection{More Logical Operators}

These operators can 

\subsubsection{$\implies$: Implication (IF-THEN)}
$p \implies q$ is false only when $p$ is true and $q$ is false.

\begin{itemize}
    \item $p$ is the \textbf{hypothesis} (or antecedent)
    \item $q$ is the \textbf{conclusion} (or consequent)
\end{itemize}

\begin{equation}
    p \implies q \equiv \lnot p \lor q
\end{equation}

\subsubsection{$\oplus$: Exclusive Disjunction (XOR)}

$p \oplus q$ is true when either $p$ is true or $q$ is true, but not both.

\begin{equation}
    p \oplus q \equiv (p \lor q) \land \lnot(p \land q)
\end{equation}

\subsection{Logical Equivalence}

Two propositions are \textbf{logically equivalent} ($\equiv$) if they have the same truth value for all possible truth assignments.

\subsubsection{De Morgan's Laws}
\begin{align}
    \neg(p \land q) &\equiv \neg p \lor \neg q \\
    \neg(p \lor q) &\equiv \neg p \land \neg q
\end{align}

\subsubsection{Double Negation}
\begin{equation}
    \neg(\neg p) \equiv p
\end{equation}

\subsubsection{Contrapositive}
\begin{equation}
    (p \implies q) \equiv (\neg q \implies \neg p)
\end{equation}

\section{First-Order Logic}

First-order logic extends propositional logic with \textbf{predicates} and \textbf{quantifiers}.

\subsection{Predicates}

A \textbf{predicate} is a statement containing variables that becomes a proposition when variables are assigned values.

Example: $P(x)$: ``$x$ is even'' becomes a proposition when $x$ is specified.

\subsection{Quantifiers}

\subsubsection{$\forall$: Universal Quantifier (FOR ALL)}
$\forall x \, P(x)$ means $P(x)$ is true for every value of $x$ in the domain.

\subsubsection{$\exists$: Existential Quantifier (THERE EXISTS)}
$\exists x \, P(x)$ means there is at least one value of $x$ in the domain for which $P(x)$ is true.

\subsection{Negating Quantifiers}

\begin{align}
    \neg(\forall x \, P(x)) &\equiv \exists x \, \neg P(x) \\
    \neg(\exists x \, P(x)) &\equiv \forall x \, \neg P(x)
\end{align}

\subsection{Multiple Quantifiers}

Order matters! $\forall x \, \exists y \, P(x, y)$ is different from $\exists y \, \forall x \, P(x, y)$.

Example: Let $P(x, y)$: ``$x < y$''
\begin{itemize}
    \item $\forall x \, \exists y \, P(x, y)$: For every number, there exists a larger number (TRUE)
    \item $\exists y \, \forall x \, P(x, y)$: There exists a number larger than all numbers (FALSE)
\end{itemize}

\chapter{Sets}

\section{What is a Set?}

A set is simply an unordered collection of unique items.

\section{Default Sets}

\subsection{$\mathbb{U}$: The Universal Set}
Contains every element in our domain (must be defined for each problem/set of problems).

\subsection{$\varnothing$: The Empty Set} 
\{\}; Has no elements.

\section{Set Operations}

\subsection{=: Equality}

Two sets are equal if all of their elements are the same.

\begin{equation}
    (A = B) \iff (\forall x (x \in A \iff x \in B))
\end{equation}

\subsection{$\vert S\vert$: Cardinality}

The cardinality of set $S$ ($\vert S \vert$) is the number of elements in set $S$.

\begin{equation}
    \vert \{1, 2, 3\} \vert = 3
\end{equation}

\subsection{$\overline{S}$: Complement}

The complement of a set $S$ ($\overline{S}$) is the set of every element (of the universal set) that is not in $S$.

For example, if $\mathbb{U} = \{1, 2, 3, 4, 5\}$ and $S = \{1, 4\}$, $\overline{S} = \{2, 3, 5\}$.

\begin{equation}
    \vert \overline{S} \vert = \vert \mathbb{U} \vert - \vert S \vert
\end{equation}

\subsection{$\cap$: Intersection}

The intersection of sets $A$ and $B$ ($A \cap B$) contains every element in both sets.

\begin{equation}
    A \cap B = \{x \vert x \in A \land x \in B\} 
\end{equation}

\subsection{$\cup$: Union}

The union of sets $A$ and $B$ ($A \cup B$) contains every element in either set (with no repeats).

\begin{align}
    A \cup B &= \{x \vert x \in A \lor x \in B\} \\
    \vert A \cup B \vert &= \vert A \vert + \vert B \vert - \vert A \cap B \vert
\end{align}

\subsection{$\subseteq$: Subset}

If $A$ is a subset of $B$ ($A \subseteq B$), all elements of $A$ are also elements of $B$.

\begin{equation}
    A \subseteq B \iff \forall x (x \in A) \implies (x \in B)
\end{equation}

Any set is a subset of itself.

\begin{equation}
    A \subseteq A
\end{equation}

The subset definition can also be used to define set equality.

\begin{equation}
    A = B \iff (A \subseteq B) \land (B \subseteq A)
\end{equation}

If $A$ is a \emph{proper} (also called strict) subset of $B$ ($A \subset B)$,
all elements of $A$ are in $B$, but $A$ is not $B$.

\begin{equation}
    A \subset B \iff (A \subseteq B) \land (A  \not= B)
\end{equation}

If $A$ is a subset of $B$, $B$ is a superset of $A$; if $A$ is a proper subset of $B$, $B$ is a proper superset of $A$.

\begin{align}
    A \subseteq B &\iff B \supseteq A \\
    A \subset B &\iff B \supset A
\end{align}

\subsection{$-$: Difference}

The difference of sets $A$ and $B$ ($A$ - $B$) is every element of $A$ that is not in $B$.

\begin{align}
    A - B &= A \cap \overline{B} \\
    \vert A - B \vert &= \vert A \cap \overline{B} \vert
\end{align}

Set difference can also be used to define a set's absolute complement (the previously defined complement):

\begin{equation}
    \overline{S} = \mathbb{U} - S
\end{equation}

The difference of sets $A$ and $B$ ($A - B$) is sometimes called the relative complement of B with respect to A ($B \backslash A$). The absolute complement of a set $S$  ($\overline{S}) $is simply its relative complement with respect to the universal set ($\mathbb{U} \backslash {S}$).

\begin{equation}
    A - B = A \backslash B
\end{equation}

The \emph{symmetric} difference of sets $A$ and $B$ ($A \Delta B)$ is every element of one set that is not in the other set.

\begin{equation}
    A \Delta B = (A - B) \cup (B - A)
\end{equation}

\section{Set Functions}

\subsection{Powerset}
\begin{itemize}
    \item input: set
    \item output: set of sets
\end{itemize}

\begin{equation}
    \mathcal{P}(S) = \{A \vert A \subseteq S\}
\end{equation}

\subsection{Cartesian Product}
\begin{itemize}
    \item input: two sets
    \item output: set of ordered pairs
\end{itemize}

\begin{equation}
    A \times B = \{(a, b) | a \in A \wedge b \in B \}
\end{equation}

If $A \cap B = \varnothing$, $A$ and $B$ are \textbf{disjoint}.

\chapter{Counting}

\begin{table}[H]
    \centering
    \begin{tabular}{ccc}
         &  No Repetition& Yes Repetition\\
         Order Matters &  Permutation $\text{P}(n, k)$& $n^k$\\
         Order Does Not Matter &  Combination $\text{C}(n, k)$ & Stars \& Bars\\
    \end{tabular}
    \caption{Which Formula To Use}
    \label{tab:placeholder}
\end{table}

\section{Counting We've Seen}

\begin{align*}
    \text{let } A &= \{1, 2, 3\} \text{ and } B = \{3, 4\}
    \\ &\therefore \\
    A \times B &= \{(1, 3), (1, 4), (2, 3), (2, 4), (3, 3), (3, 4)\}
\end{align*}


\begin{equation}
    \vert A \times B \vert = \vert A \vert \cdot \vert B \vert
\end{equation}

\begin{equation}
    \vert \mathcal{P}(S) \vert = 2^{\vert S \vert}
\end{equation}

\section{Product Rule: Order Matters}

\subsection{Two Separate Tasks}
\begin{itemize}
    \item one task in $n$ ways
    \item one task in $m$ ways
\end{itemize}

\begin{equation}
    \text{There are } n \cdot m \text{ ways to do task 1 \textbf{and} task 2.}
\end{equation}

\subsection{One Task Multiple Ways}
When repetition is okay, we have $k$ tasks and $n$ choices per task; there are $n^k$ ways to do it.

When repetition is not okay, we have $k$ tasks and $n$ choices for the first task, $n-1$ for the second task, etc. For this, we use permutations: $\text{P}(n, k)$.

\begin{align}
    \text{$\text{P}(n, k)$} &= \text{$k$ permutations of $n$ objects is an ordering of $k$ of the objects.} \\
    \text{P}(n, k) &= \frac{n!}{(n-k)!} \text{; } \text{P}(n, n) = n!
\end{align}

\section{Sum Rule: Order Does Not Matter}

\subsection{Two Tasks}
\begin{itemize}
    \item one task in $n$ ways
    \item one task in $m$ ways
\end{itemize}

\begin{equation}
    \text{There are } n + m \text{ ways to do task 1 \textbf{or} task 2.}
\end{equation}

\subsection{One Task Multiple Ways}
When repetition is okay, see the Stars and Bars subsection!

When repetition is not okay, we have $k$ tasks and $n$ choices for the first task, $n-1$ for the second task, etc. For this, we use combinations: $\text{C}(n, k)$.

\begin{align}
    \text{$\text{C}(n, k)$} &= \text{$k$ combinations of $n$ objects is a set of $k$ of the objects.} \\
    \text{C}(n, k) &= \frac{n!}{k!(n-k)!} \text{; } \text{C}(n, n) = 1
\end{align}

\subsection{Stars and Bars}

\begin{enumerate}
    \item Order doesn't matter
    \item Repetition is allowed
\end{enumerate}

How many ways there are to put $n$ indistinguishable balls into $n$ distinguishable bins?

This is secretly a combination problem! 

\begin{align}
    n_{\text{combination}} &= n_{\text{stars and bars}} + k_{\text{stars and bars}} + 1 \\
    k_{\text{combination}} &= k_{\text{stars and bars}} -1
\end{align}

\end{document}
